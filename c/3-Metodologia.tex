% ||||||||||||||||||||||||||||||||||||||||||||||
% CAPITULO 3
% ||||||||||||||||||||||||||||||||||||||||||||||
\chapter{Metodologia}\label{cap:metodologia}

	Dado todo o embasamento levantado até aqui, inicia-se a etapa da descrição do projeto, manufatura e controle da planta proposta ainda na introdução desta monografia. Este capítulo ficará dividido de forma a apresentar o modelo de cada das partes constituintes do acumulador, apresentando seu projeto e dimensionamento e todas as premissas estipuladas para montagem da planta e controle e os indicadores utilizados para atestar o funcionamento e o cumprimento dos objetivos propostos.
	
	O modelo proposto de planta contará com a montagem de um pequeno acumulador vertical de 5 rolos, com controle motorizado de deslocamento do carro móvel. Serão construídos ainda mais dois sistemas bobinadores responsáveis por desbobinar material para o acumulador e bobinar o material na saída dele, conforme pode ser visto na figura \ref{fig:plantaesboco}:
	
\begin{figure}[H]
	\caption{Esboço idealizado da planta proposta.}
	\centering
	\includegraphics[width = 0.5\linewidth]{i/plantaesboco.png}
	\label{fig:plantaesboco}
	\fonte{Elaborado pelo autor.} 
\end{figure}

% ----------------------------------------------

\section{Protótipo e Modelagem do Sistema}

	A partir desta seção será iniciado a apresentação dos modelos e dimensionamento das partes, iniciando-se pelo projeto do acumulador, seguido pelo dimensionamento e modelo típico que descreve um motor de tensão contínua e por fim o projeto e dimensionamento do desbobinador e bobinador. Para fins de simplicidade, os dois últimos serão mencionados sempre como bobinadores.

\subsection{Acumulador Vertical}

	Conforme indicado anteriormente, será exposto o projeto e dimensionamento do acumulador. Este terá como dimensões totais de $\SI{200x200x400}{{\milli\metre}}$, sendo composto por perfis de alumínio extrudados de $\SI{20x20}{{\milli\metre}}$, peças de junção modeladas e impressas em impressora 3D, um motor de tensão contínua com redutor, correias sincronizadoras, guias com rolamentos lineares, sistema de nivelamento com roldanas, mancais com rolamentos, eixos de transmissão e rolos de alumínio usinados. Na figura \ref{fig:acumuladorproj}, é apresentado o modelo 3D final do projeto desenvolvido no \textit{software} \textit{SolidWorks}\textcopyright :

\begin{figure}[H]
	\caption{Projeto mecânico do acumulador vertical com todas as partes.}
	\centering
	\includegraphics[width = 0.7\linewidth]{i/acumuladorproj.png}
	\label{fig:acumuladorproj}
	\fonte{Elaborado pelo autor.} 
\end{figure}

	Nas próximas seções serão explicadas as principais partes que compõem o acumulador, detalhando todos os pontos relevantes de seus projetos. Importante ressaltar que os cálculos de resistência mecânica e atritos da estrutura não serão abordados neste trabalho por não se tratar do foco do mesmo.

\subsubsection{Estrutura}
	
	É o corpo principal do acumulador que fornece toda a rigidez necessária para que o sistema apresente a menor quantidade de vibrações e distorções, de forma a reduzir os distúrbios externos que irão desestabilizar o sistema e resistir as forças de tração do sistema de movimentação.

\begin{figure}[H]
	\caption{Estrutura do acumulador vertical.}
	\centering
	\includegraphics[width = 0.5\linewidth]{i/acumuladorestrut.png}
	\label{fig:acumuladorestrut}
	\fonte{Elaborado pelo autor.} 
\end{figure}

	A estrutura do acumulador é formada por um conjunto de perfis de alumínio extrudado padronizado \textit{P20101} da Alu-Cek, unidas através de diversas junções plásticas impressas fixadas com parafusos \textit{Allen} DIN 912 M4x10 e porcas martelo. Por fim, é instalada uma tampa superior para futura instalação do motoredutor e sistema de movimentação do carro móvel.
	
\begin{figure}[H]
	\caption{Perfil de alumínio extrudado padronizado.}
	\centering
	\includegraphics[width = 0.3\linewidth]{i/perfilalum20x20.png}
	\label{fig:perfilalum20x20}
	\fonte{Adaptado de \citeonline{Alucek}}
\end{figure}
	
\subsubsection{Rolo Desviador}\label{cap:rolopassagem}
	
	Os rolos deviadores são de alumínio com o diâmetro de $\SI{30}{{\milli\metre}}$ com as pontas de eixo usinadas para encaixe em rolamentos rígidos de esfera \textit{SKF 608} com diâmetro interno de $\SI{8}{{\milli\metre}}$, diâmetro externo de $\SI{22}{{\milli\metre}}$, largura de $\SI{7}{{\milli\metre}}$ e momento de atrito $C_{fk} = \SI{0,0016}{{\newton\metre}}$, conforme dados de catálogo \cite{SKF}. Sobre a estrutura são montados 3 conjuntos de rolos na parte interna, formando a cama fixa de rolos, e dois conjuntos montados na parte externa do acumulador para guiar a entrada e saída do material. Com o uso do \textit{software} \textit{SolidWorks}\textcopyright , obteve-se o momento de inércia de cada rolo, sendo $J_k = \SI{397,99}{{\gram.\centi\square\metre}}$, para fins de dimensionamento será adotado $J_k = \SI{398}{{\gram.\centi\square\metre}}$.
	
\begin{figure}[H]
	\caption{Mancais e rolo desviador.}
	\centering
	\includegraphics[width = 0.6\linewidth]{i/mancalrolo.png}
	\label{fig:mancalrolo}
	\fonte{Elaborado pelo autor.} 
\end{figure}
	
\subsubsection{Carro Móvel}

	A parte mecânica mais importante do acumulador é também formada por perfis de alumínio, junções impressas, contando com rolamentos lineares mancalizados, por onde corre as guias de aço retificado de $\SI{8}{{\milli\metre}}$ e guias laterais com roldanas para evitar desalinhamentos do carro durante a movimentação. Uma peça nomeada de \textit{clamp} também é instalada lateralmente no carro com o objetivo de fixá-la à correia sincronizadora movida pelo motoredutor que transmite o movimento de elevação ao sistema. No carro são montados dois conjuntos de rolos desviadores que operam em conjunto com os três montados na parte interna estrutura, podendo ser visto na figura \ref{fig:carromovel}. Com o \textit{software} \textit{SolidWorks}\textcopyright , foi possível ter uma estimativa da massa do carro, que será necessária para a etapa de modelagem do sistema e definição do motoredutor, sendo esta de aproximadamente $\SI{1348,96}{{\gram}}$, ou, para fins de dimensionamento, $m_t = \SI{1,35}{{\kilogram}}$, logo, $m_tg \approx \SI{13,5}{{\newton}}$.
	
\begin{figure}[H]
	\caption{Carro móvel.}
	\centering
	\includegraphics[width = 0.65\linewidth]{i/carromovel.png}
	\label{fig:carromovel}
	\fonte{Elaborado pelo autor.} 
\end{figure}

\subsubsection{Modelo do Acumulador}

	Conforme abordado no referencial, nesta sessão será desenvolvido o modelo que descreve a dinâmica do acumulador a partir das equações já apresentadas e determinadas premissas de projeto. Segundo a figura \ref{fig:acumuladorsecoes}, a configuração de rolos do acumulador proposto é apresentada como um acumulador com $n = 4$ seções e $n+1 = 5$ rolos:
	
\begin{figure}[H]
	\caption{Modelo do acumulador proposto por seções.}
	\centering
	\includegraphics[width = 0.6\linewidth]{i/acumuladorsecoes.png}
	\label{fig:acumuladorsecoes}
	\fonte{Elaborado pelo autor.} 
\end{figure}
	
	Pode-se observar a existência de uma dependência de cada seção entre dois rolos consecutivos com a seção seguinte e a anterior através das tensões parciais presentes  nas seções. Se for isolado apenas um rolo e fazer o balanço dos torques que agem nele, chega-se à relação \ref{eq:velk}, já apresentada no referencial:

\begin{equation}
	\frac{d}{d_t}(J_k\omega_k) = C_{mk} - C_{rk} - C_{fk}\tag{\ref{eq:velk}}
\end{equation}

	Aqui recorda-se que o termo $C_{mk}$ representa o torque exercido pelo motor para um rolo acionado, $C_{rk}$ representa o torque equivalente exercido pelo material no rolo e $C_{fk}$ que agrega o somatório dos torques exercidos pelas forças de fricção no eixo do rolo. A partir desta equação já é possível fazer algumas simplificações, como referente ao termo $C_{mk}$ que não é aplicável uma vez que os rolos de passagem são livres, assim como o torque $C_{fk}$ também pode ser ignorado, por se tratando da mancalização dos rolos com rolamentos rígidos de esferas que apresenta forças de fricção muito menores do que a exercida pelo material, segundo visto na seção \ref{cap:rolopassagem}. Com isto, o modelo do rolo isolado pode ser representado de acordo com a figura \ref{fig:roloisolado}:

\begin{figure}[H]
	\caption{Modelo de um rolo isolado.}
	\centering
	\includegraphics[width = 0.3\linewidth]{i/roloisolado.png}
	\label{fig:roloisolado}
	\fonte{Elaborado pelo autor.} 
\end{figure}

	Equacionando o modelo e substituindo o termo $C_{rk}$ pelas trações parciais $T_k$ e $T_{k+1}$, tens-se:
	
\begin{equation}\label{eq:JOmega}
	\frac{d}{d_t}(J_k\omega_k) = R_k (T_{k+1} - T_k)
\end{equation}

	Considerando as relações de grandezas angulares e lineares, onde $\omega_k = \frac{V_k}{R_k}$, logo:
	
\begin{equation}\label{eq:ddtvk}
	\frac{d}{d_t}V_k = \frac{R_k}{J_k} R_k (T_{k+1} - T_k)
\end{equation}

	Contudo, ainda resta conhecer o valor das tensões parciais impressas pelo material. Partindo da ideia de prender o material na entrada e travar o carro em uma posição fixa, conforme a figura \ref{fig:acumuladorcarriage}, faz-se uma análise de como as tensões se comportam ao longo do acumulador.

\begin{figure}[H]
	\caption{Modelo do acumulador estático.}
	\centering
	\includegraphics[width = 0.6\linewidth]{i/acumuladorcarriage.png}
	\label{fig:acumuladorcarriage}
	\fonte{Elaborado pelo autor.} 
\end{figure}

	Uma das primeiras premissas levantadas diz sobre desconsiderar o alongamento do material, encarando-o como uma manta rígida. Ao aplicar uma força $T_o$ na saída do acumulador com o material preso, esta força deverá aparecer ao longo de todas as seções de forma a impedir a rotação dos rolos. Nesta hipótese, $T_{k+2}$ deverá ter módulo igual ao de $T_o$ de forma que no somatório dos torques sobre o rolo seja zero e assim não haja aceleração. Assim, pode-se concluir que todas as tensões apresentadas na figura \ref{fig:acumuladorcarriage} terão módulos iguais e por isso as acelerações de todos os rolos será nula. Da mesma forma, se fosse instalado um dinamômetro no carro vertical, medindo a força $F_m$, poderia se concluir que:

\begin{equation}
	F_m - m_tg - T_{k-1} - T_{k} - T_{k+1} - T_{k+2} = m_tg a_{acc}
\end{equation}

	Conforme a hipótese anterior, todas as trações observadas ao longo do material em cada seção terá o mesmo valor de $T_o$, e como o carro permanece travado não haverá aceleração, zerando o lado direito da igualdade. Assim, pode-se concluir que:
	
\begin{equation}\label{eq:Fm}
	F_m = m_tg + 4T_o
\end{equation}

	No caso em que o material seja solto na entrada e nele seja impressa uma força $T_i$, terá que a tração $T_o$ da equação \ref{eq:Fm} será substituída pela diferença entre as tensões, onde:

\begin{equation}\label{eq:Fm1}
	F_m = m_tg + 4(T_o - T_i)
\end{equation}

	Por fim, o termo $(T_o - T_i)$ na equação \ref{eq:Fm1} pode ser substituído pelo valor da tensão nominal desejada no material, que será chamada de $T_k$, e a constante $4$ deve ser entendida como o número de seções $n$ do acumulador. Assim:

\begin{equation}\label{eq:Fm2}
	F_m = m_tg + nT_k
\end{equation}

	A relação \ref{eq:Fm2} apresenta um caso específico em que o carro permanece em equilíbrio, onde o material poderia estar em repouso ou em velocidade constante, mas descarta situações em que há um desbalanceamento das forças e que para isso o carro deva se mover para compensar. Então, a equação \ref{eq:Fm2} pode ser reescrita de forma a considerar o efeito:

\begin{equation}
	F_m - (m_tg + nT_k) = m_tg a_{acc}
\end{equation}

	Se a aceleração linear do carro for substituída pelo diferencial de velocidade observado, tens-se:
	
\begin{equation}\label{eq:Fm3}
	F_m - (m_tg + nT_k) = m_tg \dot{V}_{acc}
\end{equation}

	Recordando-se da equação \ref{eq:JmWm} apresentada no referencial, que relaciona o balanceamento de torques apreciados pelo motor e sistema de movimentação:

\begin{equation}
	J_m\dot{\omega}_m = C_m - R_m(\Sigma T_k + F_{fv}(\dot{x}) \pm F_{fs} + m_tg)
	\tag{\ref{eq:JmWm}}
\end{equation} 

	Pode-se fazer pequenas adaptações e mesclar as equações \ref{eq:JmWm} e \ref{eq:Fm3}. Com a utilização de guias, rolamentos lineares e o sistema de nivelamento guiando o carro móvel, acaba por haver um contato quase desprezível dele com a estrutura, assim, as forças de fricção dinâmica e estática abordadas em \ref{eq:JmWm} como $F_{fv}(\dot{x})$ e $F_{fs}$, podem ser desconsideradas da modelagem visto a pouca influência que exercerão ao sistema. O momento de inércia $J_m$ pode ser reescrito como $J_s$ para representar a inércia de todo o sistema de movimentação e não somente do motor, evitando confusões. A velocidade angular $\dot{\omega}_m$ será reescrita como a velocidade linear de movimentação do carro $\dot{V}_{acc}$. Então:
	
\begin{equation}\label{eq:dotvacc}
	\frac{J_s}{R_s} \dot{V}_{acc} = R_s[F_m - (m_tg + nT_k)]
\end{equation}

	Onde $R_s$ representa o raio do eixo de saída do sistema de movimentação. Reorganizando os termos da equação \ref{eq:dotvacc} e aplicando a transformada de Laplace, tens-se:

\begin{equation}\label{eq:svacc}
	sV_{acc}(s) = \frac{R_s}{J_s} R_s[F_m(s) - (m_tg + nT_k(s))]
\end{equation}

	Ao aplicar a transformada de Laplace à equação \ref{eq:vacc1} e em seguida derivar a transformada, obtêm-se:

\begin{equation}\label{eq:svacc1}
	sV_{acc}(s) = \frac{s}{n}(V_{in}(s) - V_{out}(s))
\end{equation}

	Substituindo a equação \ref{eq:svacc1} em \ref{eq:svacc}, fica-se com:
	
\begin{equation}\label{eq:svacc2}
	\frac{s}{n}(V_{in}(s) - V_{out}(s)) = \frac{R_s}{J_s} R_s[F_m(s) - (m_tg + nT_k(s))]
\end{equation}	

	Após alguns ajustes algébricos, tens-se:

\begin{equation}\label{eq:svacc3}
	(V_{in}(s) - V_{out}(s)) = \frac{nR_s}{sJ_s} [\tau_m(s) - R_s(m_tg + nT_k(s))]
\end{equation}

	Onde $\tau_m(s) = R_sF_m(s)$, entendido como o torque aplicado pelo sistema de movimentação. Uma vez que $T_k$ é dado como uma variável de entrada para a tensão de interesse sobre o material, pode-se concluir que se houver um aumento de $V_{in}$ em relação à $V_{out}$ e se o carro permanecesse imóvel o valor apreciado do torque $\tau_m$ deve diminuir gradativamente, pois o material começaria a afrouxar e com isso $T_k$ também diminuiria. De forma a reverter este processo $\tau_m$ deverá aumentar, representando um deslocamento ascendente do carro do acumulador. A recíproca será verdadeira. Então, caso $V_{out}$ se sobrepunha à $V_{in}$ e o carro permaneça imóvel, o torque apreciado irá aumentar, representando um maior esticamento no material e, provavelmente, o rompimento do mesmo. De forma a evitar esta situação, $\tau_m$ deve diminuir e com isso gerar um movimento descendente do carro. Estas situações são entendidas como as situações de acúmulo e desacúmulo, respectivamente. Em equilíbrio, com as velocidades de entrada e saída iguais, conforme a equação \ref{eq:Fm2}, o torque será igual ao somatório dos torques gerados pelo material e pelo próprio peso do carro:

\begin{equation}\label{eq:taums}
	\tau_m(s) = R_s(m_tg + nT_k(s))
\end{equation}

	Do ponto de vista de um sistema com pequenas variações na posição do carro, a equação \ref{eq:taums} também é válida e servirá para desenvolver o controle do sistema.

\subsection{Bobinadores}\label{cap:bobinadores}

	A partir desta seção será apresentado o projeto mecânico do bobinador e desbobinador, que por similaridade serão chamados apenas por bobinadores. Eles terão como dimensões totais de $\SI{254x60x100}{{\milli\metre}}$, sendo composto apenas por peças impressas, um motoredutor de tensão contínua e o rolo principal terá pontas de eixo quadrado de forma a facilitar a montagem e remoção do rolo de papel. Na figura \ref{fig:Bobinadorproj}, é apresentado o modelo 3D final do projeto desenvolvido no \textit{software} \textit{SolidWorks}\textcopyright :

\begin{figure}[H]
	\caption{Projeto mecânico do bobinador com todas as partes.}
	\centering
	\includegraphics[width = 0.6\linewidth]{i/Bobinadorproj.png}
	\label{fig:Bobinadorproj}
	\fonte{Elaborado pelo autor.} 
\end{figure}

	A estrutura principal dos bobinadores é formada por dois corpos espelhados e praticamente idênticos que irão sustentar o rolo principal e o motoredutor. Assim como o acumulador, as peças são fixadas com o uso de parafusos \textit{Allen} DIN 912, podendo ser vista na figura \ref{fig:Bobinadorestrut} com mais detalhes:
	
\begin{figure}[H]
	\caption{Diagrama de forças do bobinador.}
	\centering
	\includegraphics[width = 0.5\linewidth]{i/Bobinadorestrut.png}
	\label{fig:Bobinadorestrut}
	\fonte{Elaborado pelo Autor.} 
\end{figure}

	O rolo principal do bobinador é acionado pelo motoredutor, sendo impresso com o diâmetro correto para que o rolo de papel seja bem encaixado, uma vez que o rolo principal apresenta boa resistência mecânica e está em contato com o tubo interno da bobina, evitando que a bobina deslize sobre o rolo ele. Na figura \ref{fig:Bobinadorrolo} é apresentado o rolo principal com os mancais quadrados:
	
\begin{figure}[H]
	\caption{Diagrama de forças do bobinador.}
	\centering
	\includegraphics[width = 0.5\linewidth]{i/Bobinadorrolo.png}
	\label{fig:Bobinadorrolo}
	\fonte{Elaborado pelo Autor.} 
\end{figure}
	
\subsubsection{Modelo do Bobinador}

	Para os bobinadores basta desenvolver o cálculo para apenas um deles, contudo, diferentemente do acumulador, nos bobinadores a carga é dinâmica, pois no desbobinador a medida que o material é desenrolado o volume de material diminui, o que implica na redução do diâmetro e do momento de inércia do sistema, alterando o torque verificado pelo motor. No bobinador ocorre o efeito contrário, onde a carga aumenta conforme o material é bobinado no equipamento e, por isso, aumentando o torque observado pelo motor.

\begin{figure}[H]
	\caption{Diagrama de forças do bobinador.}
	\centering
	\includegraphics[width = 0.4\linewidth]{i/bobinadormodelo.png}
	\label{fig:bobinadormodelo}
	\fonte{Elaborado pelo Autor.} 
\end{figure}

	Conforme a figura \ref{fig:bobinadormodelo}, onde $T$ é a tensão no material, $r_1$ é o menor raio da bobina, ou o raio do próprio rolo do bobinador, $r_2$ é o maior raio da bobina, $m_r$ é a massa do rolo e $m_b$ é a massa da bobina, o torque observado pelo motor depende da tensão no material e do volume de material ainda na bobina, tal que:
	
\begin{equation}
	r_2 (T - F_{mb}) = J_{bb} \dot{\omega_b} 
\end{equation}

\begin{equation}\label{eq:taubob}
	\tau_{mb} = r_2 T - J_{bb} \dot{\omega_b}
\end{equation}

	Onde $\tau_{mb}$ é o torque do motor, $J_{bb}$ é o momento de inércia do sistema e $\dot{\omega_b}$ é a derivada da velocidade angular da bobina. Contudo, a inércia é diretamente influenciada pelo volume de material e possui uma parte constante devido à inércia do rolo. Neste caso:
	
\begin{equation}
	J_{bb} = \frac{1}{2} [m_r r_1^2 + m_b(r_1^2 + r_2^2)]
\end{equation}

	No desbobinamento o motor normalmente atua como um freio para manter a tensão no material e com isso a inércia do sistema ajuda o motor a não deixar com que o material acelere. Do ponto de vista do bobinador, a inércia do sistema prejudica seu funcionamento, uma vez que ele deve manter-se sempre tracionando o material, porém, em operação normal o bobinador só assume velocidades superiores a do desbobinador se o último desacelerar, implicando que após a partida e estabilização do processo a velocidade do bobinador será sempre constante e assim desconsiderando a aceleração do sistema. Então, como pode-se perceber, durante a operação em regime com velocidade constante, a inércia do sistema não influi e pode ser desconsiderada. Resultando em:
	
\begin{equation}
	\tau_{mb} = r_2 T
\end{equation}

	Para este projeto, será construído apenas um bobinador para fazer o esforço de puxar o papel através do acumulador, mas não faz parte do escopo o controle deste bobinador, sendo acionado diretamente com velocidade constante.

\subsection{Modelo de um Motor de Tensão Contínua}\label{cap:modeomotor}

	Para os acionamentos dos sistemas de bobinamento e movimentação do acumulador foi definido o uso de motores de corrente contínua (CC). Então, como parte necessária para o desenvolvimento do controle, faz-se necessário a geração do modelo que descreve a física do motor. Segundo \citeonline{Leonhard2001}, o modelo de um motor CC pode ser definido pelo circuito de armadura e de campo que geram através de forças magnéticas o movimento do sistema mecânico do rotor, visto na figura \ref{fig:motormodel}:

\begin{figure}[H]
	\caption{Modelo de um motor de corrente contínua.}
	\centering
	\includegraphics[width = 0.6\linewidth]{i/motormodel.png}
	\label{fig:motormodel}
	\fonte{Adaptado de \citeonline{Ogata2011}.} 
\end{figure}
	
	A equação diferencial que representa o circuito de armadura do motor é definida como:
	
\begin{equation}\label{eq:motorarm}
	v_a (t) - e_a = L_a \frac{d}{dt} i_a (t) + R_a i_a (t)
\end{equation}	
	
	Onde, $v_a$ é a tensão de armadura, $e_a$ é a tensão induzida, $R_a$ é a resistência de armadura, $L_a$ é indutância de armadura e $i_a(t)$ é a corrente de armadura. Sabendo que a tensão induzida no rotor é dada como:
	
\begin{equation}\label{eq:ea}
	e_a (t) = K_{\omega} \omega (t)
\end{equation}

	Onde, $K_{\omega}$ é a constante de velocidade, pode-se substituir a equação \ref{eq:ea} na equação \ref{eq:motorarm}, obtendo-se:
	
\begin{equation}\label{eq:va}
	v_a (t) - K_{\omega} \omega (t) = L_a \frac{d}{dt} i_a (t) + R_a i_a (t)
\end{equation}	

	O torque observado no rotor do motor é visto como:

\begin{equation}\label{eq:taum}
	\tau_m (t) = K_T i_a (t)
\end{equation}	
	
	Onde, $\tau_m(t)$ é o torque do motor e $K_T$ é a constante de torque. Considerando o sistema mecânico do rotor, pode-se definir que:
	
\begin{equation}\label{eq:taum1}
	\tau_m (t) = J\frac{d}{dt} \omega (t) + B\omega (t)
\end{equation}

	Onde, $J$ representa o momento de inércia do sistema mecânico em que $J = J_m + J_s$, para $J_m$ como o momento de inércia do motor e $J_s$ o momento de inércia da carga acoplada ao eixo. O atrito viscoso do sistema mecânico é representado por $B$ em que $B = B_m + B_s$, para $B_m$ como o atrito viscoso do motor e $B_s$ o atrito viscoso da carga acoplada. Substituindo a equação \ref{eq:taum} na equação \ref{eq:taum1}, tens-se:
	
\begin{equation}\label{eq:ktia}
	K_T i_a (t) = J\frac{d}{dt} \omega (t) + B\omega (t)
\end{equation}
		
	Ainda, é possível se utilizar de recursos de notação, para apresentar as equações com $\theta$ como a posição angular do rotor, medido em radianos:

\begin{equation}
	\omega (t) = \frac{d}{dt} \theta (t) = \dot{\theta}
\end{equation}

\begin{equation}
	\alpha (t) = \frac{d}{dt} \omega (t) = \ddot{\theta}
\end{equation}

	Substituindo os recursos de notação nas equações \ref{eq:ktia} e \ref{eq:va}, chega-se em:

\begin{equation}\label{eq:va1}
	v_a (t) - K_{\omega} \dot{\theta}= L_a \dot{i_a} + R_a i_a (t)
\end{equation}

\begin{equation}\label{eq:ktia1}
	K_T i_a (t) = J \ddot{\theta} + B \dot{\theta}
\end{equation}

	Aplicando a Transformada de Laplace nas equações \ref{eq:ktia1} e \ref{eq:va1}, obtêm-se:
	
\begin{equation}\label{eq:sva}
	V_a (s) - sK_{\omega} \theta = sL_a I_a + R_a I_a (s)
\end{equation}

\begin{equation}\label{eq:sktia}
	K_T I_a (s) = s^2 J \theta + sB \theta
\end{equation}

	Isolando o termo $I_a (s)$ em ambas as equações e em seguida as igualando, tens-se:
	
\begin{equation}\label{eq:vaia}
	\frac{V_a (s) - sK_{\omega} \theta (s)}{(sL_a + R_a)} = \theta (s) \frac{s^2 J + sB}{K_T}
\end{equation}

	Ao isolar e fazer a equação \ref{eq:vaia} em função da razão $\frac{\theta(s)}{V_a (s)}$, obtêm-se a função de transferência da posição angular de um motor CC genérico:
	
\begin{equation}\label{eq:motorteta}
	\frac{\theta(s)}{V_a (s)}  = \frac{K_T}{s[s^2 J L_a + s(JR_a + L_a B) + (K_T K_{\omega} + BR_a)]}
\end{equation}

	Ao derivar a equação \ref{eq:motorteta} se obtem a função de transferência da velocidade angular do motor CC genérico:

\begin{equation}\label{eq:motormodel}
	\frac{\omega(s)}{V_a (s)} = \frac{s\theta(s)}{V_a (s)}  = \frac{K_T}{s^2 J L_a + s(JR_a + L_a B) + (K_T K_{\omega} + BR_a)}
\end{equation}

	A partir desta equação, pode-se fazer a representação em diagrama de blocos do motor conforme a figura \ref{fig:motorbloco}:
	
\begin{figure}[H]
	\caption{Representação em diagrama de blocos de um motor CC.}
	\centering
	\includegraphics[width = 1\linewidth]{i/motorbloco.png}
	\label{fig:motorbloco}
	\fonte{Elaborado pelo Autor.} 
\end{figure}

	Como percebe-se no diagrama, não há a possibilidade de se acessar diretamente algumas variáveis intermediárias como a corrente e torque do motor. Neste caso, é feita uma nova representação do diagrama de blocos a partir do diagrama de fluxo de sinal, usando as equações \ref{eq:sva} e \ref{eq:sktia} \cite{Nise2012}:
	
\begin{figure}[H]
	\caption{Representação em diagrama de blocos explodido de um motor CC.}
	\centering
	\includegraphics[width = 1\linewidth]{i/motorblocos.png}
	\label{fig:motorblocos}
	\fonte{Elaborado pelo Autor.} 
\end{figure}

	Agora com o diagrama da figura \ref{fig:motorblocos} pode-se acessar diretamente a leitura de corrente e velocidade, além da aplicação de torque ao eixo do motor, visto como $T_d$.

\subsection{Material de Teste}

	Nesta sessão é descrito brevemente as características mecânicas do material a ser utilizado nos testes da planta. Então, a primeira definição a ser feita diz respeito às dificuldades encontradas para se obter um material elastomérico além da complexidade de controlar um material imprevisível no sistema. Por isso, decidiu-se por optar por um material de fácil acesso, que apesar de apresentar características muito distintas, para os fins de modelagem e controle da planta será suficiente para o desenvolvimento do projeto. O material escolhido para operação será um rolo de papel Mili Bianco Folha Simples de 60 metros.
	
	Os dados técnicos disponíveis sobre as características mecânicas do tipo de papel escolhido são escassos e para o desenvolvimento e dimensionamento do projeto serão considerados os valores apresentados na tabela \ref{tab:caracpapel}, conforme fornecidos por contato informal por meio de correio eletrônico com um funcionário da Mili.

\begin{table}[H]
  \caption{Características mecânicas do material de teste}
  \label{tab:caracpapel}
  \centering%
  \begin{minipage}{.53\textwidth}
    \begin{tabular*}{\textwidth}{cc}
      \hline
      Grandeza & Unidade \\ \hline
      \hline
      Resistência à Tração [$\sigma_r$]  &  $\SI{90}{\newton\per\metre}$ \\ 
      Tensão Nominal [$T_{ro}$]   &  $\SI{12}{\newton}$ \\
      Alongamento Máximo [$\Delta l_r$]   &  $\SI{30}{\milli\metre}$  \\
      Densidade de Massa [$\rho_r$]   &  $\SIrange{0,15}{0,22}{\gram\per\cubic\metre}$  \\
      Massa Aproximada [$m_r$]   &  $\SI{106}{\gram}$  \\
      Diâmetro Interno [$D_{ri}$]   &  $\SI{46}{\milli\metre}$  \\
      Diâmetro Externo [$D_{ro}$]   &  $\SI{112}{\milli\metre}$  \\
      Largura do Rolo [$L_r$]   &  $\SI{100}{\milli\metre}$  \\
      Espessura Aproximada [$e_r$]   &  $\SIrange{0,08}{0,11}{\milli\metre}$  \\ \hline
    \end{tabular*}
    \fonte{Cortesia de \citeonline{Fagundes2019}.} 
  \end{minipage}
\end{table}

	Recordando que estes dados não estão disponíveis para acesso público e serão utilizados para fins de dimensionamento inicial e posteriormente ajustados conforme a dinâmica identificada na planta.

\subsection{Dimensionamento dos Motores}

	Com os dados obtidos do sistema mecânico projetado e das características mecânicas do material de teste, torna-se possível calcular e dimensionar a motorização utilizado na planta. Então, ao utilizar-se da equação \ref{eq:taums} e considerando que a variação de velocidade do carro sempre será pequena, o peso morto do sistema é de $\SI{13,5}{\newton}$, a tensão nominal do material é de $\SI{12}{\newton}$ conforme tabela \ref{tab:caracpapel}, e sabendo que o leito da polia sincronizadora por onde a correia corre tem diâmetro de $\SI{12}{\milli\metre}$, calcula-se o torque mínimo que o sistema de movimentação deve entregar para o sistema mecânico:
	
\begin{equation}
	\tau_m = \frac{12}{2}(13,5 + 4 \cdot 12) = \SI{369}{\newton \cdot \milli\metre}
\end{equation}

	Alterando as unidades, tens-se:
	
\begin{equation}
	\tau_m = \SI{3690}{\gram f \cdot \centi\metre}
\end{equation}

	Como o sistema de transmissão conta com um redutor de $1:64$, o torque sentido pelo motor deve ser de aproximadamente $\SI{57,6}{\gram f \cdot \centi\metre}$. Então, com base no torque necessário para movimentar o carro móvel, o modelo de motor CC escolhido é o \textit{AK360/78.8PL12S7000S} da Neoyama. As características do motor podem ser vistas na tabela \ref{tab:caracmotor}:

\begin{table}[H]
  \caption{Características do motor AK360/78.8PL12S7000S.}
  \label{tab:caracmotor}
  \centering%
  \begin{minipage}{.4\textwidth}
    \begin{tabular*}{\textwidth}{cc}
      \hline
      Grandeza & Unidade \\ \hline
      \hline
      Tensão Nominal	&	$\SI{12}{{\volt}}$ \\
      Faixa de Tensão	&	$\SIrange{6}{24}{\volt}$ \\ 
      Rotação a Vazio	&	$\SI{7000}{RPM}$  \\
      Corrente a Vazio	&	$\SI{170}{\milli\ampere}$  \\
      Rotação com Carga	&	$\SI{5700}{RPM}$  \\ 
      Corrente Máxima	&	$\SI{590}{\milli\ampere}$  \\
      Torque Máximo		&	$\SI{78,8}{\gram f \cdot \centi\metre}$  \\
      Potência Máxima	&	$\SI{4,58}{\watt}$  \\
      Rendimento		&	$\SI{63}{\percent}$  \\ 
      Torque de Partida	&	$\SI{407}{\kilogram f \cdot \centi\metre}$  \\ \hline
    \end{tabular*}
    \fonte{Adaptado de \citeonline{AK555b}.} 
  \end{minipage}
\end{table}

	Através da tabela \ref{tab:caracmotor}, verifica-se que o torque nominal do motor é de $\SI{78,8}{\gram f \cdot \centi\metre}$ e a rotação a plena carga do motor é de $\SI{5700}{RPM}$, logo, com a redução de $1:64$, espera-se na saída um torque de $\SI{5043}{\gram f \cdot \centi\metre}$ e uma rotação de $\SI{89,06}{RPM}$. Se considerar um fator de perda de $20\%$ no redutor, tens-se um torque de $\SI{4034,4}{\gram f \cdot \centi\metre}$ no eixo de saída a uma velocidade linear máxima de:
	
\begin{equation}
	V_{acc} = \frac{\pi D_p \eta_a}{60} = \frac{\pi \cdot 12 \cdot 89,06}{60} \approx 56 mm/s
\end{equation}

	Onde, $D_p$ é o diâmetro do leito da polia sincronizadora, $\eta_a$ é rotação máxima na saída do redutor e $60$ é o fator de conversão de minutos para segundos.

	Para os bobinadores, conforme visto na seção \ref{cap:bobinadores}, considerando a equação \ref{eq:taubob} e desprezando o termo dependente da aceleração, pode-se calcular os valores máximo e mínimo de torque exigidos pelo sistema usando os valores de diâmetro do rolo de acordo com a tabela \ref{tab:caracpapel}, onde:
	
\begin{equation}
	\tau_{mbmax} = r_2 T = 5,6 \cdot 1200 = \SI{6720}{\gram f \cdot \centi\metre}
\end{equation}

\begin{equation}
	\tau_{mbmin} = r_2 T = 2,3 \cdot 1200 = \SI{2760}{\gram f \cdot \centi\metre}
\end{equation}

	Foi visto que a velocidade máxima que o carro móvel pode alcançar com a redução é de $56 mm/s$ e pela equação \ref{eq:vacc1} que a velocidade do carro móvel depende da diferença da velocidade de entrada e de saída e do número de seções do acumulador. Então, seguindo pela premissa que a maior diferença de velocidade se dá quando a velocidade de entrada está no máximo e a velocidade de saída no mínimo, reajustando os termos pode-se definir que:
	
\begin{equation}
	V_{in} = nV_{acc} + V_{out}
\end{equation}

	Para o valor de $V_{out} = 0$, tens-se:
	
\begin{equation}
	V_{in} = 4 \cdot 56 = 224 mm/s
\end{equation}

	Da mesma forma como o torque varia com raio, a rotação do bobinador também deve varia para manter a mesma velocidade linear do material, assim:

\begin{equation}
	\eta_{mbmax} = \frac{60 V_{in}}{2 \pi r_2} = \frac{60 \cdot 224}{2 \pi 23} = \SI{93}{RPM}
\end{equation}
	
\begin{equation}
	\eta_{mbmin} = \frac{60 V_{in}}{2 \pi r_2} = \frac{60 \cdot 224}{2 \pi 56} = \SI{38,2}{RPM}
\end{equation}

	Então, com base no torque e na rotação calculados, para os bobinadores os motores escolhidos são do modelo \textit{AK360/78.8PL12S7000S}, o mesmo modelo usado no acumulador, mantendo apenas uma referência de motor e redutor para todo o projeto. Apesar da redução na rotação máxima, não será um problema para o sistema, apenas influindo na redução da velocidade máxima de movimentação do acumulador.

% ---------------------------------------------- 
    
\section{Acionamento e Instrumentação}

	Nesta seção está descrito o hardware empregado na automação e medição dos sistemas do protótipo proposto. Primeiramente é abordado os motores escolhidos e o projeto de acionamento deles, seguindo pelo método utilizado para medir o torque dos motores e a posição do carro vertical e finalizando com a medição de velocidade dos bobinadores.

\subsection{Acionamento dos Motores}

	Motores de corrente contínua são largamente aplicados em diversas aplicações eletrônicas, desde uma simples furadeira elétrica até trens. Os tipos mais comuns de motores CC são os chamados motores \textit{brushed} ou motores com escovas. Devido a enorme gama de aplicações, este tipo de motor também precisa ser controlado de forma a garantir as exigências do processo. Desta forma, o motor de corrente contínua possui uma grande vantagem devido a linearidade da sua relação tensão-velocidade, o que possibilita várias formas de controle, como: um potenciômetro em série com o motor, um transistor na região ativa com controle da corrente de base e o uso de um controlador PWM (\textit{Pulse Width Modulation} - Modulação de Largura de Pulso) \cite{Hughes2006}.
	
	Apesar do método de controle mais simples ser com o potenciômetro em série, há um grande desperdício de energia uma vez que a tensão que não é imposta às bobinas do motor ficam no potenciômetro que irá dissipar esta energia em forma de calor, tornando-se um sistema extremamente ineficiente, apesar de eficaz. O mesmo pode ser dito do controle com um transistor na região ativa, já que igualmente ao caso anterior, toda a energia não dissipada pelo motor será convertida em calor pelo transistor. Neste caso, a opção com maior aplicação e rendimento é o controle PWM, onde um transistor em série com o motor atua somente nas regiões de corte e saturação onde a dissipação de potência é mínima. O transistor recebe um \textit{clock} variável na base com frequência fixa, em que é alterado a largura do pulso de sinal alto variando de $\SI{0}{\percent}$ até $\SI{100}{\percent}$. Este percentual de tempo com sinal alto é conhecido como \textit{Duty Cyle}, sendo representado por \cite{Ozer2017}:
	
\begin{equation}\label{eq:duty}
	DC_{\%} = \frac{T_{ON}}{T} \times 100
\end{equation}

	Na equação \ref{eq:duty}, para simplificação de notação o \textit{Duty Cycle} será chamado por $DC_{\%}$, o termo $T_{ON}$ é o tempo de sinal alto e $T$ é o período da frequência do PWM, ambos medidos em segundos. 
	
\begin{equation}\label{eq:fpwm}
	f_{PWM} = \frac{1}{T}
\end{equation}

	A medida que o valor do $DC_{\%}$ aumenta até $\SI{100}{\percent}$, a tensão média enxergada nos terminais do motor se aproxima da tensão da fonte de alimentação apenas reduzida pela queda de tensão $V_{CE}$ no transistor em série. Isto é melhor entendido na figura \ref{fig:PWMGraph}:

\begin{figure}[H]
	\caption{Representação do sinal PWM nos terminais de um motor CC.}
	\centering
	\includegraphics[width = 0.63\linewidth]{i/PWMGraph.png}
	\label{fig:PWMGraph}
	\fonte{Adaptado de \citeonline{Ozer2017}.} 
\end{figure}

	De acordo com a figura anterior, percebe-se que com o aumento do \textit{Duty Cycle}, a tensão média aumenta e isto se deve a este parâmetro ser em função da área abaixo da curva do sinal PWM.
	
\begin{equation}
	V_{med} = \frac{1}{T}\int_{0}^{T_{ON}} V_s dt
\end{equation}

	Onde $V_{med}$ é tensão média vista nos terminais do motor e $V_s$ é a amplitude da tensão aplicada na entrada do controlador.

	Contudo, o método aplicado com um único transistor em série com o motor não permite a operação bidirecional, isto é, nesta topologia não é possível alterar a direção para qual o eixo do motor irá girar. Com isto, foi criada a topologia de Ponte H, também conhecida como \textit{H-Bridge}, que recebe este nome pois o sistema é basicamente composto por quatro transistores dispostos de tal forma a criar duas linhas verticais e paralelas onde o motor é colocado entre as linhas conectando-as dando a aparência de uma letra "H" para a montagem, conforme figura \ref{fig:HBridgeFull}.

\begin{figure}[H]
	\caption{Esquema de uma ponte H com representação da alteração de sentido de rotação.}
	\centering
	\includegraphics[width = 0.63\linewidth]{i/HBridgeFull.png}
	\label{fig:HBridgeFull}
	\fonte{Adaptado de \citeonline{Ozer2017}.} 
\end{figure}

	Com este tipo de topologia, os transistores são acionados aos pares e de lados opostos da ponte de forma a fechar um circuito em "S" para a corrente fluir pelo motor fazendo-o girar para um sentido. Quando se deseja alterar o sentido de rotação, basta trocar o par acionado e o motor será energizado com a polaridade invertida fazendo com que o sentido mude \cite{ROHM2009}. As pontes H podem ser projetadas conforme a necessidade do projetista, é possível adquirir componentes integrados (CI) em que a ponte H já vem embarcada e ou comprar módulos prontos em que é apenas necessário alimentar e mandar o sinal PWM.
	
	Considerando que o objetivo desta monografia não é o projeto eletrônico do sistema, decidiu-se por adquirir um módulo pronto para acionamento de até dois motores CC ou um motor de passo, visto na figura \ref{fig:L298N}:

\begin{figure}[H]
	\caption{Driver com dupla ponte H completa para acionamento de motores CC.}
	\centering
	\includegraphics[width = 0.3\linewidth]{i/L298N.png}
	\label{fig:L298N}
	\fonte{Adaptado de \citeonline{L298N}.} 
\end{figure}

	Este módulo vem equipado com o CI principal \textit{L298} de fabricação da \textit{STMicroeletronics} que é um integrado com duas pontes H completas e quatro entradas digitais para o comando PWM e inversão de sentido dos motores. O esquema de pinagem do módulo é observado na figura \ref{fig:SchL298N}: 

\begin{figure}[H]
	\caption{Esquema de ligação do driver L298N elaborado na ferramenta \textit{Fritzing}\textcopyright.}
	\centering
	\includegraphics[width = 0.4\linewidth]{i/SchL298N.png}
	\label{fig:SchL298N}
	\fonte{Elaborado pelo Autor.} 
\end{figure}

	De acordo com a figura, observa-se a simplicidade da ligação do módulo, que disponibiliza os pinos das entradas digitais para controle. Inclusive, a partir de três \textit{jumpers} é possível habilitar e desabilitar os canais de saída e o circuito regulador de tensão do módulo, onde se o último estiver desabilitado é necessário fornecer a tensão de controle externamente.
	
	Serão necessários dois módulos para contemplar os dois motores do projeto e assim é possível a concentração dos esforços sobre o entendimento do sistema e da elaboração do controle sem a necessidade de se preocupar com implementação e comissionamento de uma etapa de eletrônica de potência.

\subsection{Medição de Corrente}

	Conforme observado na seção \ref{cap:modeomotor}, o torque verificado no eixo do motor depende diretamente da constante de torque do motor e da corrente de armadura. Com isso, significa que ao medir-se a corrente do motor e sabendo o valor de $K_T$ é possível calcular o torque $\tau_m$ e assim fazer o controle de torque do motor e, indiretamente, o controle de tensão do material. Para se fazer medição de corrente de motores, comumente usa-se resistores sensores  que são resistores de baixíssima resistência em série com a carga. Então, mede-se a queda de tensão sobre ele, desta forma, sabendo o valor da resistência e a queda de tensão calcula-se a corrente que circula no circuito através da Lei de Ohm:
	
\begin{equation}
	I = \frac{V}{R}
\end{equation}

	Para este projeto, optou-se pela aplicação de um módulo baseado no CI \textit{INA219} de fabricação da \textit{Texas Instruments} que é um sensor bidirecional de corrente e monitor de energia com uma interface de comunicação \textit{I2C} com capacidade de retornar informações de corrente, tensão e potência na carga com uma acurácia máxima de $\SI{0,5}{\percent}$ em uma faixa de tensão de $\SIrange{0}{26}{\volt}$ \citeonline{INA219a}. Na figura \ref{fig:INA219} é visto o módulo de medição:

\begin{figure}[H]
	\caption{Módulo bidirecional de monitoração de corrente por resistor \textit{shunt}.}
	\centering
	\includegraphics[width = 0.3\linewidth]{i/INA219.png}
	\label{fig:INA219}
	\fonte{Adaptado de \citeonline{INA219}.} 
\end{figure}

\subsection{Medição de Tensão}

	Apesar da possibilidade de controle indireto de tensão no material, optou-se por utilizar um método de medição direta com o uso de \textit{strain gauges}, que são sensores resistivos para medição da distância de deformação de corpos, assim podendo obter informações como força, pressão e aceleração atuantes sobre o corpo medido. Os  \textit{strain gauges} são extremamente úteis no campo de análise de estresse nos mais diversos tipos de corpos, funcionando por um princípio descoberto por William Thompson em 1856, de que fios de cobre ou de ferro tinham a sua resistência elétrica alterada quando eram esticados ou comprimidos por alguma força externa \citeonline{Strain2009}.
	
	Graças a esta descoberto, hoje os \textit{strain gauges} são largamente utilizados na criação de células de carga, inclusive sendo aplicados em linhas de processamento com sistemas de acúmulo para medição da tensão do material e controle em malha fechada \citeonline{Gilbert2018}. Para isto, foi instalado nos mancais centrais da cama fixa de rolos, sensores \textit{strain gauges} para a medição direta da deformação do mancal, causada pelo esforço do papel sendo tracionado pelo carro móvel, conforme figura \ref{fig:acumuladorstrain}.

\begin{figure}[H]
	\caption{Célula de carga e medição de tensão no acumulador.}
	\centering
	\includegraphics[width = 0.6\linewidth]{i/acumuladorstrain.png}
	\label{fig:acumuladorstrain}
	\fonte{Elaborado pelo autor.} 
\end{figure}

	Conforme a figura \ref{fig:acumuladorstrain}, com o sensor montado nesta posição, é possível fazer a medição direta da tensão no material, sendo esta duas vezes maior, devido ao laço de papel que envolve o rolo de alumínio. Para este projeto, optou-se pela utilização de um módulo baseado no CI \textit{HX711} de fabricação da \textit{Mouser Electronics} que é um conversor analógico digital (\textit{ADC}) de 24 bits para medição de células de carga. Na figura \ref{fig:HX711} é visto o módulo de medição:

\begin{figure}[H]
	\caption{Módulo conversor de célula de carga.}
	\centering
	\includegraphics[width = 0.3\linewidth]{i/HX711.png}
	\label{fig:HX711}
	\fonte{Adaptado de \citeonline{HX711}.} 
\end{figure}


\subsection{Medição de Posição e Velocidade}

	Outra realimentação importante para o sistema é a posição onde o carro móvel está, isso pois há um limite tanto para a posição mínima como máxima que ele pode atingir devido às limitações mecânicas do sistema. Para isto, imagina-se utilizar um sensor de posição no carro e aplicando uma derivada à variação de posição medida é possível calcular a velocidade instantânea. Existem diversas formas diferentes de se fazer a medição de posição e velocidade de um motor desde métodos sem uso de sensor (\textit{sensorless}) ao uso de diversos tipos de sensores diferentes como óticos e ultrassônicos \cite{Dwivedi2017}.

\begin{figure}[H]
	\caption{Módulo de medição de velocidade por pulsos.}
	\centering
	\includegraphics[width = 0.3\linewidth]{i/KY-040.png}
	\label{fig:KY-040}
	\fonte{Adaptado de \citeonline{KY-040}.} 
\end{figure}

	Para a fazer as medições decidiu-se pelo uso de um módulo de medição rotativo por pulsos de fabricação \textit{Keyes} que é um sensor muito parecido com um potenciômetro, visto na figura \ref{fig:KY-040}, que possui resolução de 20 pulsos por volta e interface de comunicação a partir de dois canais A e B que enviam um trem de pulsos em código \textit{Gray}, conforme o eixo do sensor é rotacionado. Neste módulo, um sensor ótico emite uma luz infravermelha para um receptor que toda vez que este feixe é cortado por uma roda perfurada emite pulsos na saída. Este tipo de funcionamento é o princípio de um \textit{encoder} padrão de mercado, que é um dispositivo de medição de velocidade/posição ótica utilizado largamente na indústria. Para poder medir corretamente a velocidade de rotação de um eixo, por exemplo, é necessário construir uma roda perfurada, onde o comprimento do arco formado entre um furo e outro será a medida do deslocamento angular por pulso. Somando a quantidade de pulsos por um intervalo de amostragem é possível obter-se a velocidade e posição angular do eixo \cite{Joshi2014}.
	
\begin{figure}[H]
	\caption{Representação de um encoder ótico}
	\centering
	\includegraphics[width = 0.4\linewidth]{i/encoder.png}
	\label{fig:encoder}
	\fonte{Elaborado pelo Autor.} 
\end{figure}

% ----------------------------------------------
    
\section{Controle}

	Nesta seção, é elaborado uma explicação rápida do tipo de controlador embarcado utilizado para este sistema, a representação da topologia empregada e a definição dos pontos chaves na elaboração do software e supervisório para todo o controle da planta.

\subsection{Embarcado}

	Procurando uma plataforma simples para desenvolvimento com uma comunidade ativa e com bastante bibliografia disponível, optou-se pela utilização de um Arduino Mega 2560 WiFi, de fabricação da \textit{RobotDyn}, que possui já integrado na placa o chip ESP8266 da \textit{Espressif}. O Arduino Mega é uma das placas de desenvolvimento mais comuns no mercado e altamente empregada em projetos de impressoras 3D e robótica caseira que conta com um microcontrolador embarcado ATmega 2560. Este CI proporciona até $54$ pinos de I/O (\textit{Inputs/Outputs} - Entradas/Saídas) digitais dos quais $15$ deles podem ser parametrizados para operar como saídas PWM e ainda conta com $16$ entradas analógicas de $10 bits$ e \textit{timer} RTC (\textit{Real Time Clock} - Relógio de Tempo Real). A placa pode ser vista na figura \ref{fig:Mega2560} a seguir, sendo observado que ela possui conexão de alimentação externa de $\SI{12}{\volt}$ e comunicação Serial/USB, além da antena para conexão WiFi.

\begin{figure}[H]
	\caption{Placa de programação Arduino Mega com microcontrolador ATMega 2560.}
	\centering
	\includegraphics[width = 0.5\linewidth]{i/Mega2560.png}
	\label{fig:Mega2560}
	\fonte{Adaptado de \citeonline{Mega2019}.} 
\end{figure}	

	Devido a alta densidade de pontos da placa, torna-se fácil a interligação de todos os sensores e desenvolvimento dos controles planejados que serão melhor exemplificados na seção seguinte. Na figura abaixo é representado simplificadamente a topologia do hardware empregado e a comunicação Serial/USB usada para enviar os comandos via supervisório para a planta, \textit{setpoint} de tensão e posição do acumulador e a alteração dos parâmetros da planta para possíveis ajustes:

\begin{figure}[H]
	\caption{Topologia de hardware empregada para o desenvolvimento do sistema.}
	\centering
	\includegraphics[width = 0.8\linewidth]{i/Topologia.png}
	\label{fig:Topologia}
	\fonte{Elaborado pelo Autor.} 
\end{figure}

\subsection{Software}

	Devido a expansão do universo de projetos caseiros com plataformas \textit{OpenSource}, a comunidade do Arduino é extremamente ativa e possui uma grande gama de funções e blocos de códigos prontos disponíveis na rede para serem usados e adaptados conforme a necessidade do programador. A interface IDE (\textit{Integrated Development Environment} - Ambiente de Desenvolvimento Integrado) utilizado para a programação é o \textit{Visual Studio Code}\textcopyright que possui uma interface bastante amigável e com suporte a diversas linguagens de programação, inclusive linguagem C e C++, com diversas bibliotecas e suporte a vários modelos de placas de desenvolvimento, como o próprio Arduino Mega. O que facilita a parametrização do hardware sem necessidade de profundo conhecimento de programação de microcontroladores. Por sua vez, isto também opera como uma desvantagem já que o uso destas bibliotecas aumenta o uso de processamento do dispositivo e assim prejudica a otimização para sistemas mais refinados que buscam trabalhar no limite da capacidade.
	
	O objetivo do software empregado no sistema é fazer periodicamente a leitura dos sensores, calcular o novo erro do sistema de controle, recalcular a malha de controle PID e atualizar as saídas PWM com um novo valor de \textit{Duty Cycle}, recebendo comandos do sistema supervisório em tempos específicos com uma interrupção de tempo.

\begin{figure}[H]
	\caption{Fluxo macro do \textit{software} de controle.}
	\centering
	\includegraphics[width = 0.9\linewidth]{i/fluxobasico.png}
	\label{fig:fluxobasico}
	\fonte{Elaborado pelo Autor.} 
\end{figure}

	O bobinador de saída operará isoladamente do sistema de controle, sendo controlado manualmente com aplicação direta de tensão em seus terminais. Enquanto que na entrada do acumulador o papel será desenrolado e freado manualmente para controlar a variação de tensão no material e assim permitir que o controle do acumulador atue e possa ser analisado e validado. O acumulador possui a opção de duas malhas de controle diferentes, sendo uma malha de posição, vista na figura \ref{fig:ctrlacumulador}, onde $P_{SP}$ é a referência de posição do carro móvel e $P_{PV}$ é a posição efetiva do carro móvel.

\begin{figure}[H]
	\caption{Diagrama de blocos da malha de posição do acumulador.}
	\centering
	\includegraphics[width = 0.9\linewidth]{i/ctrlacumulador.PNG}
	\label{fig:ctrlacumulador}
	\fonte{Elaborado pelo Autor.} 
\end{figure}

	Já a segunda malha de controle é uma malha de tensão, vista na figura \ref{fig:ctrlacumulador1}, onde $T_{SP}$ é a referência de tensão no material sendo esticado e $T_{PV}$ é a tensão efetiva medida pela célula de carga.

\begin{figure}[H]
	\caption{Diagrama de blocos da malha de tensão do acumulador.}
	\centering
	\includegraphics[width = 0.9\linewidth]{i/ctrlacumulador1.PNG}
	\label{fig:ctrlacumulador1}
	\fonte{Elaborado pelo Autor.} 
\end{figure}

	O sistema supervisório é desenvolvido em linguagem $C\sharp$ no ambiente do \textit{VisualStudio}\textcopyright  e servirá para alterar os valores de referência do sistema, como tensão e posição, além dos ganhos $K_p$, $K_i$ e $K_d$ das malhas de controle para eventuais ajustes manuais. A partir dele também é possível visualizar o valor das principais variáveis de processo como corrente do motor, tensão na célula de carga, velocidade e posição do carro móvel. Através do supervisório será possível selecionar o tipo de modo de controle como manual, por tensão ou por posição, além de gerar relatórios do funcionamento da planta.
	
	O método de sintonia escolhido para encontrar os parâmetros da planta e parametrizar o controlador PID é o método do relé com histerese já comentado neste trabalho. Desta forma, colocando o carro móvel a oscilar sem carga e lendo a frequência e amplitude da oscilação, pode-se obter os ganhos do controlador utilizando a equação \ref{eq:n(a)} e o quadro \ref{qua:ZieglerNichols}, assim tendo um ponto de partida para ajustar e otimizar o sistema com outras formas de controle mais robustas e refinadas, sendo que a função relé pode ser executada através do sistema supervisório.

% ----------------------------------------------
















