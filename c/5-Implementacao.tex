% ||||||||||||||||||||||||||||||||||||||||||||||
% CAPITULO 5
% ||||||||||||||||||||||||||||||||||||||||||||||
\chapter{Implementação do Controle em Tempo Discreto}

	\lipsum[1]
	
%	\begin{lstlisting}[caption={Minimização da função do controlador PID}]
%funcao(Kp, Ti, Td, n, dados_objetivo, y_original, soma_erro) :
%    erro = referencia - dados_objetivo[n]
%    se (n>0) :
%        soma_erro += erro
%        derivada =  ({}dados_objetivo[n] -  dados_objetivo[n-1]/dt
%        saida_controlador = Kp * erro + (Kp/Ti)*soma_erro*dt - Kp*Td*derivada
%    se nao :
%        saida_controlador = 0
%    n += 1
%    retorna saida_controlador - y_original
%\end{lstlisting}
	
% ----------------------------------------------
	
\section{Organização e Estrutura do Software}

	\lipsum[1]

\subsection{Software Embarcado}

	\lipsum[1]
	
\subsection{Software Supervisório}

	\lipsum[1]
	
% ----------------------------------------------	

\section{Comportamento com Controle Proporcional}

	\lipsum[1]

% ----------------------------------------------

\section{Comportamento com Controle Proporcional-Integral}

	\lipsum[1]
	
% ----------------------------------------------

\section{Comportamento com Controle Proporcional-Integral-Derivativo}

	\lipsum[1]

% ----------------------------------------------

\section{Comparação entre os Controladores}

	\lipsum[1]

% ----------------------------------------------