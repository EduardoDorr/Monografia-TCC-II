% ||||||||||||||||||||||||||||||||||||||||||||||
% CAPITULO 1 - INTRODUÇÃO
% ||||||||||||||||||||||||||||||||||||||||||||||
\chapter{Introdução}\label{Introdução}

% ----------------------------------------------

	Desde muitos anos a necessidade de sistemas industriais mais inteligentes, capazes de se auto-regularem e tomar decisões de forma objetiva e eficiente não necessitando de contínuas intervenções humanas vem se tornando uma necessidade mais eminente e trivial. O advento da chamada Indústria 4.0, conceito esse inicialmente apresentado na feira de Hannover em 2011 \cite{SEBRAE2016}, trouxe uma grande quebra de paradigma na forma como vinha se fazendo automação, e assim apresentando uma topologia de descentralização do controle dos diversos processos com dispositivos inteligentes capazes de se comunicarem entre si, trocando informações, corrigindo distorções, se adaptando e aprendendo.
	
	Naturalmente, vários gestores e engenheiros começaram a ver grandes oportunidades de aumentar a eficiência de suas fábricas, desde ganhos diretos de produção como diagnóstico rápido, diminuição do tempo de manutenção, capacidade de ver as variáveis de processo, eficiência e disponibilidade das máquinas de qualquer local, inclusive na tela de seus próprios smartphones, e tudo isso em tempo real graças a enorme integração entre a automação e a tecnologia da informação \cite{FIA2018}.
	
	Não obstante, a indústria da borracha também não ficaria para trás e também vem exigindo profundas melhorias e atualização dos principais equipamentos utilizados neste setor que já remontam mais de um século e ainda são utilizados hoje com pouquíssimas diferenças em seus sistemas de controle \cite{Groover2012}. Grande parte dos poucos avanços proferidos se dá também pelas propriedades do principal composto desta indústria, a borracha, que não possui comportamento linear, pois em processo possui propriedades de um fluído não newtoniano \cite{Mark2013}.

	A indústria da borracha, possui uma grande gama de oportunidades de inovação de forma a aprimorar o controle, sensoriamento e inteligência das máquinas. Contudo, para isso é necessário um vasto conhecimento dos processos e entendimento da importância que os sistemas de controle têm neste caminho. %Sendo assim, será que os alunos formados nos cursos de engenharia com foco em controle e automação saem do ambiente universitário preparados e com uma boa noção da realidade dos processos e sistemas de controle aplicados na indústria, salvo os alunos que já operam na área?
	
	Entre os diversos desafios encontrados durante a graduação, sempre surgiu uma necessidade latente de aplicação prática dos conteúdos discorridos, assim relacionando toda a teoria aprendida na construção de soluções para problemas reais vistos na indústria. A decorrência desta necessidade fomentou a elaboração deste trabalho, visando desenvolver um projeto didático e miniaturizado de um sistema real encontrado em diversas linhas de máquinas nos setores de produção de papel, polímeros, metais e elastômeros.
	
	Com o desenvolvimento deste sistema, serão relacionadas diversas áreas de conhecimento que foram abordadas na graduação, tais como teoria de sistemas de controle, eletrônica de potência, programação, DSP e física, além de conhecimentos a parte necessários para a completa formação de um engenheiro, como entendimento dos processos de fabricação, projeto de máquinas e a percepção de como todo engenheiro, independente da sua área específica de atuação, precisa ter noções mínimas das demais áreas para poder desenvolver suas atividades.
	
	O objeto de estudo se limitará ao projeto mecânico, projeto eletrônico, modelagem matemática, manufatura e comissionamento, levantamento dos dados da planta, desenvolvimento do algoritmo de controle e implementação em um controlador embarcado de uma pequena linha de bobinamento e desbobinamento de papel com um acumulador vertical, que recria uma etapa de um grande sistema de processamento de mantas, carcaças e correias de borracha. A utilização do papel como material, deve-se a facilidade de obtenção, uma vez que a topologia de controle seria igual para quaisquer dos materiais supracitados, apenas interferindo nos parâmetros da planta e do controle.
	
	No decorrer do trabalho, será apresentado brevemente o funcionamento da indústria da borracha e onde este tipo de equipamento se insere no ramo, além de explanar sobre a teoria de sistemas de controle, métodos de sintonização de controladores, seguido da apresentação de toda a modelagem matemática, projeto da maquete e a solução proposta para implementar e testar o sistema.

% ----------------------------------------------

\section{Objetivos}

	Todos os conhecimentos desenvolvidos se tornarão ferramentas poderosas de trabalho na vida profissional do engenheiro, e assim, espera-se que uma maquete didática com capacidade de desafiar e promover o crescimento de futuros discentes dos cursos de controle das engenharias da Unisinos, venha a aguçar e proporcionar toda a experiência prática envolvida na escolha da abordagem e solução de um desafio real.
	
	Assim, objetiva-se alcançar os seguintes pontos:
	
\begin{enumerate}[label=\alph*)]
	\item Conhecer uma importante etapa de um processo de produção comum na indústria;
	\item Compreender a base física e matemática aplicada na modelagem de um sistema;
	\item Simular e analisar o comportamento de um sistema a partir da modelagem do mesmo;
	\item Construir uma maquete didática que simula em pequena escala uma planta real;
	\item Analisar os parâmetros da planta e comparar com o modelo calculado;
	\item Desenvolver um controlador PID discreto em que possa selecionar entre uma malha de controle de tensão do material a partir de sensores \textit{strain gauge}, ou uma malha de controle de posição do carro móvel a partir de um encoder;
	\item Analisar os resultados obtidos no controle simulado da planta;
	\item Implementar o controlador desenvolvido em um sistema embarcado;
	\item Analisar os resultados obtidos no controle implementado da planta, comparando com os resultados simulados.
\end{enumerate}

% ----------------------------------------------