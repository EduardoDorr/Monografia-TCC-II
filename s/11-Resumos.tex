% ----------------------------------------------
% Resumo em português
% ----------------------------------------------
% Importante: De acordo com a NBR6024 as palavras-chaves devem ser separadas entre si por ponto e devem ter somente a primeira palavra escrita com letra maiúscula
\setlength{\absparsep}{18pt} % ajusta o espaçamento dos parágrafos do resumo
\begin{resumo}

	As indústrias de polímeros, elastômeros, papéis e metais operam há várias décadas utilizando equipamentos similares para criar pulmões nas linhas contínuas de produção de forma a evitar paralisações de processo. Contudo, um grande desafio é controlar a tensão sobre o material de forma a mantê-la homogênea durante o processo, principalmente em etapas de acúmulo e desacumulo do material que provocam distúrbios de tensão que se propagam por toda a linha de produção, podendo prejudicar o produto final. Portanto, o presente trabalho apresenta uma linha miniaturizada de bobinamento de papel com um acumulador vertical, com o objetivo de controlar a tensão sobre o material ou a posição do carro móvel, aplicando os conhecimentos clássicos de controle e desenvolvendo um controlador PID digital com função de sintonia, embarcado em uma placa de prototipação Arduino e com sistema de supervisão serial. Onde o acumulador apresentou resultados satisfatórios, sendo controlável e versátil, favorecendo seu uso como uma ferramenta de estudos para a área de controle e automação.
	    
	\vspace{\onelineskip}
	 
	\noindent 
	\textbf{Palavras-chaves}: Controladores PID. acumulador vertical de mantas. bobinador. controle de posição. controle de tensão. arduino. supervisório.
\end{resumo}

% ----------------------------------------------
% Resumo em inglês
% ----------------------------------------------
% Importante: De acordo com a NBR6024 as palavras-chaves devem ser separadas entre si por ponto e devem ter somente a primeira palavra escrita com letra maiúscula
%\begin{resumo}[Abstract]
%\begin{otherlanguage*}{english}
%
%	\lipsum[7]
%    
%	\vspace{\onelineskip}
% 
%	\noindent 
%	\textbf{Key-words}: Microelectromechanical systems (MEMS). Electronic insulation. Switching power supply. Isolated power supply. Electric isolated circuitry. Low cost power supply.
%\end{otherlanguage*}
%\end{resumo}

% ----------------------------------------------
% resumo em francês 
% ----------------------------------------------
% Importante: De acordo com a NBR6024 as palavras-chaves devem ser separadas entre si por ponto e devem ter somente a primeira palavra escrita com letra maiúscula
% \begin{resumo}[Résumé]
%  \begin{otherlanguage*}{french}
%     Il s'agit d'un résumé en français.
 
%    \textbf{Mots-clés}: latex. abntex. publication de textes.
%  \end{otherlanguage*}
% \end{resumo}

% ----------------------------------------------
% resumo em espanhol
% ----------------------------------------------
% Importante: De acordo com a NBR6024 as palavras-chaves devem ser separadas entre si por ponto e devem ter somente a primeira palavra escrita com letra maiúscula
% \begin{resumo}[Resumen]
%  \begin{otherlanguage*}{spanish}
%    Este es el resumen en español.
  
%    \textbf{Palabras clave}: latex. abntex. publicación de textos.
%  \end{otherlanguage*}
 
% \end{resumo}